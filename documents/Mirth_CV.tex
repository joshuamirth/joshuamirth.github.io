%% start of file `template.tex'.
%% Copyright 2006-2013 Xavier Danaux (xdanaux@gmail.com).
%
% This work may be distributed and/or modified under the
% conditions of the LaTeX Project Public License version 1.3c,
% available at http://www.latex-project.org/lppl/.


\documentclass[11pt,letterpaper,sans]{moderncv}        % possible options include font size ('10pt', '11pt' and '12pt'), paper size ('a4paper', 'letterpaper', 'a5paper', 'legalpaper', 'executivepaper' and 'landscape') and font family ('sans' and 'roman')

% moderncv themes
\moderncvstyle{classic}                             % style options are 'casual' (default), 'classic', 'oldstyle' and 'banking'
\moderncvcolor{green}                               % color options 'blue' (default), 'orange', 'green', 'red', 'purple', 'grey' and 'black'
%\renewcommand{\familydefault}{\sfdefault}         % to set the default font; use '\sfdefault' for the default sans serif font, '\rmdefault' for the default roman one, or any tex font name
%\nopagenumbers{}                                  % uncomment to suppress automatic page numbering for CVs longer than one page

% character encoding
%\usepackage[utf8]{inputenc}                       % if you are not using xelatex ou lualatex, replace by the encoding you are using
%\usepackage{CJKutf8}                              % if you need to use CJK to typeset your resume in Chinese, Japanese or Korean

% adjust the page margins
\usepackage[margin=1in]{geometry}
\usepackage{nicefrac}
%\usepackage{hyperref}
%\setlength{\hintscolumnwidth}{3cm}                % if you want to change the width of the column with the dates
%\setlength{\makecvtitlenamewidth}{10cm}           % for the 'classic' style, if you want to force the width allocated to your name and avoid line breaks. be careful though, the length is normally calculated to avoid any overlap with your personal info; use this at your own typographical risks...

% personal data
\name{Joshua}{Mirth}
\title{Curriculum Vitae}                               % optional, remove / comment the line if not wanted
%\address{Colorado State University \\ Department of Mathematics \\ 1874 Campus Delivery \\ Fort Collins, Colorado \\ Office: Weber 232}% optional, remove / comment the line if not wanted; the "postcode city" and "country" arguments can be omitted or provided empty
%\phone[fixed]{(970) 491-5284}                   % optional, remove / comment the line if not wanted; the optional "type" of the phone can be "mobile" (default), "fixed" or "fax"
%\phone[fixed]{+2~(345)~678~901}
%\phone[fax]{+3~(456)~789~012}
\email{mirth@math.colostate.edu}                               % optional, remove / comment the line if not wanted
\homepage{www.math.colostate.edu/\textasciitilde mirth/}                         % optional, remove / comment the line if not wanted
%\social[linkedin]{john.doe}                        % optional, remove / comment the line if not wanted
%\social[twitter]{jdoe}                             % optional, remove / comment the line if not wanted
%\social[github]{CrossXProduct}                              % optional, remove / comment the line if not wanted
%\extrainfo{additional information}                 % optional, remove / comment the line if not wanted
%\photo[64pt][0.4pt]{picture}                       % optional, remove / comment the line if not wanted; '64pt' is the height the picture must be resized to, 0.4pt is the thickness of the frame around it (put it to 0pt for no frame) and 'picture' is the name of the picture file
%\quote{``A mathematician is a device for turning coffee into theorems.''}                                 % optional, remove / comment the line if not wanted

% to show numerical labels in the bibliography (default is to show no labels); only useful if you make citations in your resume
%\makeatletter
%\renewcommand*{\bibliographyitemlabel}{\@biblabel{\arabic{enumiv}}}
%\makeatother
%\renewcommand*{\bibliographyitemlabel}{[\arabic{enumiv}]}% CONSIDER REPLACING THE ABOVE BY THIS

% bibliography with mutiple entries
\usepackage{multibib}
\newcites{book,misc}{{Books},{Others}}
%----------------------------------------------------------------------------------
%            content
%----------------------------------------------------------------------------------
\begin{document}
%\begin{CJK*}{UTF8}{gbsn}                          % to typeset your resume in Chinese using CJK
%-----       resume       ---------------------------------------------------------
\makecvtitle

\section{Academic Employment}
\cventry{Beginning Fall 2020}{Postdoctoral Research Associate}{}{Michigan State University}{Department of Computational Mathematics, Science, and Engineering}{}

%\vspace{-.5cm}
\section{Education}
\cventry{2015--2020}{Doctorate}{}{Colorado State University}{\emph{Mathematics}}{Thesis -- \emph{Vietoris--Rips Metric Thickenings and Wasserstein Spaces.} Advisor: Henry Adams}  % arguments 3 to 6 can be left empty
\cventry{2015--2017}{Master of Science}{}{Colorado State University}{\textit{Mathematics}}{Thesis -- \emph{Metric Thickenings of Euclidean Submanifolds}. Advisor: Henry Adams.}
\cventry{2011--2015}{Bachelor of Science}{Summa Cum Laude, with Departmental Honors}{Hillsdale College}{\textit{Mathematics}}{Senior Thesis -- \emph{Functional Analysis and the Dirichlet Problem}, minor in physics. Advisor: David Gaebler.}

\section{Publications}
\cvitem{Submitted}{\emph{Operations on Metric Thickenings} with Henry Adams and Johnathan Bush.}
\cvitem{2020}{``A torus model for optical flow.'' With Henry Adams, Johnathan Bush, Brittany Carr, and Lara Kassab. \emph{Pattern Recognition Letters} 129 (2020), 304-310. Conference version, ``On the nonlinear statistics of optical flow,'' \emph{Proceedings of Computational Topology in Image Context}, LNCS Volume 11382 (2019), 151-165. Available at arXiv:1812.00875.}
\cvitem{2019}{``Metric thickenings of Euclidean submanifolds'' with Henry Adams. \emph{Topology and its Applications} 254 (2019), 69-84. Available at arXiv:1709.02492.}
\cvitem{2019}{\emph{A fractal dimension for measures via persistent homology} with Henry Adams, Manuchehr Aminian, Elin Farnell, Michael Kirby, Rachel Neville, Chris Peterson, Patrick Shipman, and Clayton Shonkwiler. To appear in \emph{Topological Data Analysis -- The Abel Symposia}, 2020. Available at arXiv:1808:01079.}

\section{Talks and Presentations}
\subsection{Research Talks}
\cvitem{2020 Apr.}{\emph{Algebraic Topology in Chemistry}, Greenslopes seminar, Colorado State University.}
\cvitem{2020 Jan.}{\emph{Morse Theory for Wasserstein Spaces}, Joint Mathematics Meetings, Denver, Colorado.}
\cvitem{2019 Jul.}{\emph{Morse Theory for Wasserstein Spaces}, Young Topologists Metting, \'{E}cole Polytechnique F\'ed\'erale de Lausanne.}
\cvitem{2019 Apr.}{\emph{Metric Spaces in Applied Topology}, Regional Workshop in Qualitative Geometry \& Topology, The Ohio State University.}
\cvitem{2018 Oct.}{\emph{Nonlinear Statistics of Optical Flow}, SPAMlab, Colorado State University.}
\cvitem{2018 Apr.}{\emph{Metric Thickenings of Euclidean Submanifolds}, Graduate Student Topology and Geometry Conference, University of Chicago (upcoming).}
\cvitem{2017 Sep.}{\emph{Metric Thickenings of Euclidean Submanifolds}, SIAM Central States Sectional Meeting,  Applied Algebraic Topology session, Colorado State University.}
\cvitem{2017 Jul.}{\emph{Metric Thickenings of Euclidean Submanifolds}, TDA: Theory and Applications, workshop at Macalaster College (Poster presentation).}
\cvitem{2015 Apr.}{\emph{Functional Analysis and the Dirichlet Problem}, Michigan Undergraduate Mathematics Conference, Hope College.}
\cvitem{2013 Jul.}{\emph{Simulating Post-Reconnection Coronal Flux Tubes} American Astronomical Society Solar Physics Division Meeting (Poster with Dana Longcope).}

%\subsection{Expository Talks}
%\cvitem{2018 Jul.}{\emph{The Yoneda Lemma}, CSU category theory seminar.}
%\cvitem{2018 Jun.}{\emph{Simplicial Sets}, CSU category theory seminar.}
%\cvitem{2017 Dec.}{\emph{Morse Theory: An Introduction}, CSU Greenslopes seminar.}

\section{Teaching}
\cventry{2015--2020}{Graduate Teaching Assistant}{Colorado State University}{Mathematics Department}{}{Instructor of record: %Taught recitation sessions, assisted in lectures, helped write and grade exams, held office hours, developed instructional materials for computer lab sessions.
%\newline{}%
%Detailed achievements:%
\begin{itemize}%
\item Math 340 -- Introduction to Ordinary Differential Equations, Spring 2018, Fall 2018, Spring 2019, Fall 2019
\item Math 261 -- Calculus for Physical Scientists III, Fall 2017
\item Math 160 --  Calculus for Physical Scientists I, Fall 2016, Spring 2017
%\item Achievement 2, with sub-achievements:
%  \begin{itemize}%
%  \item Sub-achievement (a);
%  \item Sub-achievement (b), with sub-sub-achievements (don't do this!);
%    \begin{itemize}
%    \item Sub-sub-achievement i;
%    \item Sub-sub-achievement ii;
%    \item Sub-sub-achievement iii;
%    \end{itemize}
%  \item Sub-achievement (c);
%  \end{itemize}
%\item Achievement 3.
\end{itemize}
Online Course Facilitator:
\begin{itemize}
\item Math 141 -- Calculus for Management Sciences, Summer 2019 (Online)
\end{itemize}
Teaching assistant:
\begin{itemize}
\item Math 161 -- Calculus for Physical Scientists II, Fall 2015, Sprint 2016 
\end{itemize}
Outreach:
\begin{itemize}
\item Co-taught (with Henry Adams) a two week course on Applied and Computational Topology at the Universidad de Costa Rica,
Summer 2017.
\end{itemize}
}%\section{Master Thesis}
%\cvitem{Title}{\emph{Metric Thickenings of Euclidean Submanifols}}
%\cvitem{Advisor}{Henry Adams}
%\cvitem{Description}{Short thesis abstract}

\section{Experience}
%\subsection{Teaching}

\subsection{Computational}
\cventry{2016--2017}{Programmer}{Colorado State University}{Environmental Health Department}{}{Developed tools for analysis of motion tracker data in \textsc{MatLab}.}
\cventry{2013}{REU}{Montana State University}{Solar Physics}{}{Developed and tested numerical models of magnetic reconnection in the solar corona. }%\newline{}Description line 2}

\subsection{Miscellaneous}
%\cventry{year--year}{Job title}{Employer}{City}{}{Description}
\cventry{2018}{Co-organizer}{Greenslopes Seminar}{Colorado State University}{}{}
\cventry{2017-2018}{Sectretary}{SIAM}{Colorado State University Student Chapter}{}{}
\cventry{2016-2017}{Treasurer}{SIAM}{Colorado State University}{}{}
\cventry{2014-2015}{Vice-President}{Kappa Mu Epsilon}{Hillsdale College Chapter}{}{}
\cventry{2013-2014}{Treasurer}{Kappa Mu Epsilon}{Hillsdale College Chapter}{}{}
\cventry{2013-2015}{Putnam Team}{}{Hillsdale College}{}{}

%\section{Languages}
%\cvitemwithcomment{Language 1}{Skill level}{Comment}
%\cvitemwithcomment{Language 2}{Skill level}{Comment}
%\cvitemwithcomment{Language 3}{Skill level}{Comment}

%\section{Computer skills}
%\cvitem{Programming}{Python, IDL, C, Java}
%\cvitem{Scientific}{MatLab, Mathematica}
%\cvitem{Other}{\LaTeX, HTML, Photo and video editing}
%
%\section{Interests}
%\cvlistitem{Craft coffee -- roasting, brewing, and exploring coffees from around the world.}
%\cvlistitem{Running -- collegiate track and cross country runner.}
%\cvlistitem{Biking -- both road and mountain}

\section{Awards}
\cvlistitem{Outstanding Graduate Teaching Assistant - Colorado State University, 2018-2019}
\cvlistitem{Taylor Award - Highest GPA among Hillsdale College Mathematics graduates (2015)}
\cvlistitem{Kimball Medal - top male athlete at Hillsdale College (2015).}
\cvlistitem{Hillsdale College Dean's List (8 semesters)}
\cvlistitem{National Merit Scholar (2011)}
\cvlistitem{NCAA Division II All-American -- three times (track and field, cross country)}
\cvlistitem{GLIAC conference champion -- four times (indoor track)}


%\section{Extra 2}
%\cvlistdoubleitem{Item 1}{Item 4}
%\cvlistdoubleitem{Item 2}{Item 5\cite{book1}}
%\cvlistdoubleitem{Item 3}{Item 6. Like item 3 in the single column list before, this item is particularly long to wrap over several lines.}
%
%\section{References}
%\begin{cvcolumns}
%  \cvcolumn{Category 1}{\begin{itemize}\item Person 1\item Person 2\item Person 3\end{itemize}}
%  \cvcolumn{Category 2}{Amongst others:\begin{itemize}\item Person 1, and\item Person 2\end{itemize}(more upon request)}
%  \cvcolumn[0.5]{All the rest \& some more}{\textit{That} person, and \textbf{those} also (all available upon request).}
%\end{cvcolumns}

% Publications from a BibTeX file without multibib
%  for numerical labels: \renewcommand{\bibliographyitemlabel}{\@biblabel{\arabic{enumiv}}}% CONSIDER MERGING WITH PREAMBLE PART
%  to redefine the heading string ("Publications"): \renewcommand{\refname}{Articles}
%\nocite{*}
%\bibliographystyle{plain}
%\bibliography{publications}                       % 'publications' is the name of a BibTeX file

% Publications from a BibTeX file using the multibib package
%\section{Publications}
%\nocitebook{book1,book2}
%\bibliographystylebook{plain}
%\bibliographybook{publications}                   % 'publications' is the name of a BibTeX file
%\nocitemisc{misc1,misc2,misc3}
%\bibliographystylemisc{plain}
%\bibliographymisc{publications}                   % 'publications' is the name of a BibTeX file

%\clearpage
%-----       letter       ---------------------------------------------------------
% recipient data
%\recipient{Some Conferecne}{Some University\\123 somestreet\\some city}
%\recipient{Tutorial on Multiparameter Persistence, Computation, and Applications}{}
%\recipient{Theory and Foundations of TGDA Workshop}{}
%\date{\today}
%\opening{Personal Statement}
%\closing{Sincerely,}
%\enclosure[Attached]{curriculum vit\ae{}}          % use an optional argument to use a string other than "Enclosure", or redefine \enclname
%\makelettertitle
%%
%Personal statement describing scientific interests, research plans, and reasons for wishing to participate.



%Personal statement describing scientific interests, research plans, and reasons for wishing to participate.

%I am a third-year doctoral student at Colorado State University working in the field of applied algebraic topology. 
%My advisor is Henry Adams.
%My research interests are in the structure of simplicial complexes (such as the Vietoris--Rips and \v{C}ech complexes arising in applied topology), discrete Morse theory, and persistent homology.
%In the fall of 2017 I completed a master's thesis on the homotopy types of metric thickenings (that is, a metric analogue of a Vietoris--Rips or \v{C}ech complex) of Euclidean submanifolds.
%Some current research projects include further work on metric thickenings---now with the vertex set being a geodesic space---as well as a collaborative, computational project with the Colorado State University Pattern Analysis Lab to define and compute a persistent homology fractal dimension.
%A long-term project is the development of a Morse theory for filtrations of (possibly infinite) simplicial complexes.
%
%Multiparameter persistence is closely related to the above topics, particularly to Morse theory.
%Similar to how multiparameter persistence studies a space equipped with several filtrations and seeks to understand the persistent homology as both vary, I am interested in a Morse theory that can identify both critical simplices, as in the established discrete Morse theory, and also critical scale parameters, as in the classical smooth setting.
%Not only am I interested in better understanding multiparameter persistence generally, but I am hopeful there may be enlightening connections here.
%I am also interested in developing a more thorough knowledge of the state of the art with regard to computational tools for persistent homology. 
%I have used Ripser and JavaPlex extensively, and have a passing familiary with RIVET, GUDHI, and some other packages, but would like to become more proficient with these and other tools.
%
%% For Ohio thing: 
%My interest in the Theory and Foundation of TGDA workshop springs from several sources.
%I have some familiarity with Dr. Matthew Kahle's work on random simplicial complexes and on Dr. Facundo Memoli's work on Gromov-Hausdorff distances, and would like to learn more.
%I am also interested in developing a more thorough knowledge of the state of the art with regard to computational tools for persistent homology. 
%I have used Ripser and JavaPlex extensively, and have a passing familiary with RIVET, GUDHI, and some other packages, but would like to become more proficient with these and other tools.
%Overall, I would like to better-understand the algorithmic and statistical aspects of applied topology, and am excited about the opportunity to develop connections within the topological and geometric data analysis community.
%


%I am interested in developing a better understanding both of this work and of the algorithmic and statistical aspects of applied topology, and am excited about the opportunity to develop connections within the topological and geometric data analysis community.
%%\makeletterclosing
%
%%\clearpage\end{CJK*}                              % if you are typesetting your resume in Chinese using CJK; the \clearpage is required for fancyhdr to work correctly with CJK, though it kills the page numbering by making \lastpage undefined
\end{document}


%% end of file `template.tex'.
